\documentclass[letterpaper,11pt]{article}
%\documentclass[12pt]{article}
%\usepackage[margin=0.5in]{geometry}
%\usepackage{hyperref}
%\usepackage{tikz}
%\usepackage{amsmath}
%\newcommand{\superscript}[1]{\ensuremath{^{\textrm{#1}}}}
%\newcommand{\unit}[1]{\ensuremath{\, \mathrm{#1}}}
%\newcommand{\inlinecode}{\texttt}

\begin{document}

\title{CS 830: Program 4}
\date{April 10, 2013}
\author{Carmen St.\ Jean}

\maketitle

\begin{enumerate}
\item \emph{Describe any implementation choices you made that you felt were important. Mention anything else that we should know when evaluating your program.}

Answer
\item \emph{What can you say about the the time and space complexity of your
program?}

Let $a$ be the number of actions available and $s$ be the number of states available in a reinforcement learning problem.

Random will have a constant time complexity, since actions are selected at random.  Expected value or utility is never calculated.  Space complexity is constant as well since no information is stored and learned.

Greedy will have a space complexity of $O(s^2 * a)$, since the transition function must be kept in memory.

Value iteration will have a space complexity of $O(s^2 * a)$ because each ($s$, $a$, $s'$) triplet must be represented in memory.  It's time complexity for each iteration will be $O(s^2 * a)$, though this partially depends on the discount factor.  A larger discount factor, near 1, will make decisions be based on results further and further in the future, so the number of iterations could grow exponentially.


\item \emph{What suggestions do you have for improving this assignment in the future?}

Answer
\end{enumerate}

\end{document}
