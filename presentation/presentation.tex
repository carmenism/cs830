\documentclass[style=unh]{powerdot}
\pdsetup{lf={Carmen St.\ Jean (University of New Hampshire)},
	 rf={}}

\usepackage{graphicx}
\usepackage{multirow}
\usepackage{amsfonts}
\usepackage{color}
\usepackage{listings}
\usepackage{colortbl}
\usepackage[thinlines]{easytable}

\newcommand{\nvs}{\vspace{-15pt}}

\title{Pattern Database Heuristics for Greedy Search}

\author{Carmen St.\ Jean\vspace{0.2in}
  \vspace{.2in} \\
}
\date{\mbox{}}

\begin{document}

\maketitle


%%%%%%%%%%%%%%%%%%%%%%%%%%%%%%%%%%%%%%%%%%%%%%%%%%%%%%%%%%%%%%%%%%%%%%%%%%%%%%%

\begin{slide}{Proposal}

Greedy search will solve the sliding tile puzzle better with the fringe pattern database than a more specialized pattern database as a heuristic. 

\end{slide}


%%%%%%%%%%%%%%%%%%%%%%%%%%%%%%%%%%%%%%%%%%%%%%%%%%%%%%%%%%%%%%%%%%%%%%%%%%%%%%%

\begin{slide}{Algorithm: Greedy Search}

Assume $b$ is branching factor and $m$ is maximum depth.

\begin{itemize}
    \item Best-first search with queue ordered by heuristic value 
    \item Complete in finite spaces
    \item Inadmissible
    \item $b^m$ time
    \item $b^m$ space
\pause{
    \item \textbf{Tends to yield suboptimal solutions in a reasonable time}}

    
\end{itemize}

\end{slide}

%%%%%%%%%%%%%%%%%%%%%%%%%%%%%%%%%%%%%%%%%%%%%%%%%%%%%%%%%%%%%%%%%%%%%%%%%%%%%%%

\begin{slide}{Domain: $N$-Puzzle}

\hspace{1in} \includegraphics[width=1.75in]{puzzle.eps}
%\vspace{.15in}

\begin{itemize}
    \item $N$ numbered square tiles, one blank tile
    \item Objective: rearrange tiles by sliding the blank space to reach goal configuration
    \item Commonly solved with A* using Manhattan Distance heuristic
\end{itemize}

\end{slide} 

%%%%%%%%%%%%%%%%%%%%%%%%%%%%%%%%%%%%%%%%%%%%%%%%%%%%%%%%%%%%%%%%%%%%%%%%%%%%%%%

\begin{slide}{Pattern Databases}

\begin{itemize}
\item A \textit{pattern} is a partial specification of a state 
\pause{\begin{itemize}
    \item Some elements of the state are abstracted
    \end{itemize}}
\pause{\item A \textit{target pattern} is a partial specification of the goal state}
\pause{\item A \textit{pattern database} is set of all patterns obtained by permuting the target pattern}
\pause{\item Every pattern knows its exact solution cost for the target pattern}
    \begin{itemize}
    \pause{\item Admissible heuristic}
    \end{itemize}
\end{itemize}

\end{slide} 

%%%%%%%%%%%%%%%%%%%%%%%%%%%%%%%%%%%%%%%%%%%%%%%%%%%%%%%%%%%%%%%%%%%%%%%%%%%%%%%


\begin{slide}{$N$-Puzzle PDBs}

\begin{itemize}
  \item Abstract away some tiles
 \pause{\item Can be more accurate than Manhattan distance}
 \pause{\item Can use multiple disjoint pattern databases at once}
 \pause{\item Lots of different abstractions possible}
     \begin{itemize}
    \pause{\item Fringe abstraction (outer edge)
    \item Special abstraction (keep tiles furthest from goal position)}
    \end{itemize}
  \pause{\item Less tiles abstracted, more powerful pattern database}    
  \pause{\item More timely to calculate and larger space required when fewer tiles are abstracted}
 
\end{itemize}

\end{slide}

%%%%%%%%%%%%%%%%%%%%%%%%%%%%%%%%%%%%%%%%%%%%%%%%%%%%%%%%%%%%%%%%%%%%%%%%%%%%%%%

\begin{slide}{Fringe Abstraction}

Independent of start configuration.\\

\pause{
\begin{TAB}(e,0.65cm,0.65cm){|c|c|c|c|}{|c|c|c|}
    & 1 & 2 & 3   \\ 
  4 & 5 & 6 & 7   \\ 
  8 & 9 & 10 & 11 \\ 
\end{TAB}
}%\vspace{.15in}
\pause{
\begin{TAB}(e,0.65cm,0.65cm){|c|c|c|c|}{|c|c|c|}
    & A & \textcolor{red}{2} & \textcolor{red}{3}  \\
  A & A & \textcolor{red}{6} & \textcolor{red}{7}   \\
  A & A & A & A \\ 
\end{TAB} %\vspace{.15in}
\begin{TAB}(e,0.65cm,0.65cm){|c|c|c|c|}{|c|c|c|}
    & A & A & A  \\
  A & A & A & A   \\
  \textcolor{blue}{8} & \textcolor{blue}{9} & \textcolor{blue}{10} & \textcolor{blue}{11} \\ 
\end{TAB}
}\\

\pause{
 Example: \\
 \begin{TAB}(e,0.65cm,0.65cm){|c|c|c|c|}{|c|c|c|}
  2  & 6 & 3 & 4  \\ 
  10 & 9 & 5 &    \\ 
  8  & 7 & 1 & 11 \\ 
\end{TAB} 
}
\pause{%\vspace{.15in}
\begin{TAB}(e,0.65cm,0.65cm){|c|c|c|c|}{|c|c|c|}
  \textcolor{red}{2}  & \textcolor{red}{6} & \textcolor{red}{3} & A  \\ 
  A & A & A &    \\ 
  A & \textcolor{red}{7} & A & A \\ 
\end{TAB}
 \begin{TAB}(e,0.65cm,0.65cm){|c|c|c|c|}{|c|c|c|}
  A  & A & A & A  \\ 
  \textcolor{blue}{10} & \textcolor{blue}{9} & A &    \\ 
  \textcolor{blue}{8}  & A & A & \textcolor{blue}{11} \\ 
\end{TAB}  
}\\




\vspace{.15in}
\pause{
Heuristic value of state:\\
    $h = \textcolor{red}{cost} + \textcolor{blue}{cost}$
}

\end{slide}


%%%%%%%%%%%%%%%%%%%%%%%%%%%%%%%%%%%%%%%%%%%%%%%%%%%%%%%%%%%%%%%%%%%%%%%%%%%%%%%

\begin{slide}{Special Abstraction}

Created for a specific start configuration, using the eight tiles furthest from their goal positions.\\
\vspace{.05in}
\pause{
 \begin{TAB}(e,0.7cm,0.7cm){|c|c|c|c|}{|c|c|c|}
  2  & 6 & 3 & 4  \\ 
  10 & 9 & 5 &    \\ 
  8  & 7 & 1 & 11 \\ 
\end{TAB} 
}%\vspace{.05in}

\vspace{.05in}

\pause{
 \begin{TAB}(e,0.7cm,0.7cm){|c|c|c|c|}{|c|c|c|}
  $2_2$  & $6_2$ & $3_1$ & $4_4$  \\ 
  $10_3$ & $9_1$ & $5_1$ &    \\ 
  $8_0$  & $7_3$ & $1_3$ & $11_0$ \\ 
\end{TAB} }
\pause{
 \begin{TAB}(e,0.7cm,0.7cm){|c|c|c|c|}{|c|c|c|}
  A  & A & A & \textcolor{red}{4}  \\ 
  \textcolor{red}{10} & A & A &    \\ 
  A  & \textcolor{red}{7} & \textcolor{red}{1} & A \\ 
\end{TAB} %\vspace{.05in}
 \begin{TAB}(e,0.7cm,0.7cm){|c|c|c|c|}{|c|c|c|}
  \textcolor{blue}{2}  & \textcolor{blue}{6} & \textcolor{blue}{3} & A  \\ 
  A & A & \textcolor{blue}{5} &    \\ 
  A  & A & A & A \\ 
\end{TAB} 
}

\vspace{.05in}

\pause{
 \begin{TAB}(e,0.7cm,0.7cm){|c|c|c|c|}{|c|c|c|}
     & 1 & 2 & 3  \\ 
  4 & 5 & 6 & 7  \\ 
  8  & 9 & 10 & 11 \\ 
\end{TAB}}
\pause{
 \begin{TAB}(e,0.7cm,0.7cm){|c|c|c|c|}{|c|c|c|}
     & \textcolor{red}{1} & A & A  \\ 
  \textcolor{red}{4} & A & A & \textcolor{red}{7}  \\ 
  A  & A & \textcolor{red}{10} & A \\ 
\end{TAB} 
 \begin{TAB}(e,0.7cm,0.7cm){|c|c|c|c|}{|c|c|c|}
     & A & \textcolor{blue}{2} & \textcolor{blue}{3}  \\ 
  A & \textcolor{blue}{5} & \textcolor{blue}{6} & A  \\ 
  A  & A & A & A \\ 
\end{TAB} 
}

\end{slide}

%%%%%%%%%%%%%%%%%%%%%%%%%%%%%%%%%%%%%%%%%%%%%%%%%%%%%%%%%%%%%%%%%%%%%%%%%%%%%%%


\begin{slide}{Insight into Proposal}

Greedy search will solve the sliding tile puzzle better with the fringe pattern database than a more specialized pattern database as a heuristic. \newline

\pause{Why?}\\

\begin{itemize}
   \pause{\item Fringe}
         \begin{itemize}
         \pause{\item Solving remaining tiles will not disturb solved tiles}          
         \pause{\item $h = 0$ means you're actually close to the goal}
         \end{itemize}
   \pause{\item Specialized}
         \begin{itemize}
         \pause{\item Solving remaining tiles might disturb solved tiles}
         \pause{\item $h = 0$ does not guarantee you're close to the goal}
          \end{itemize}
\end{itemize}

\end{slide}


%%%%%%%%%%%%%%%%%%%%%%%%%%%%%%%%%%%%%%%%%%%%%%%%%%%%%%%%%%%%%%%%%%%%%%%%%%%%%%%


\begin{slide}{Results}

\begin{table}
\begin{tabular}{ r | c | c }
              & Greedy & A* \\ \hline
  Fringe      & 44,171 &  218,816 \\
  Specialized & 196,073 &  333,928 \\
\end{tabular}
\caption{Number Nodes Expanded}
\end{table}

\pause{This shows:
\begin{itemize}
    \item Fringe is better than specialized for both greedy and A* on puzzles of this size
    \pause\item Previously, it was shown specialized works better on big puzzles with big pattern databases, this does not generalize to smaller puzzles with disjoint pattern databases \\ \pause{(Which one?)}
\end{itemize}}

\end{slide}

\end{document}
